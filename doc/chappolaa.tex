\chapter{Numerical details}\label{chappola}
\markboth{\uppercase{Numerical details}}{\uppercase{Numerical details}}

\minitoc

\section{Polarizability}


The DDA discretizes the object into a set of punctual dipoles, where a
polarizability $\alpha$ is associated to each punctual dipoles. There
are different forms for this polarizability. The first to have been
used, and the simplest, is the relation of Clausius Mossotti
(CM)~\cite{Purcell_AJ_73}:
%%%%%%%%%%%%%%%%%%%%%%%%%%%%%%%%%%%%%%%%%%%%%%%%%%%%%%%%%
\be \alpha_{\rm CM} & = & \frac{3}{4\pi}
\frac{\varepsilon-1}{\varepsilon+2}d^3=
\frac{\varepsilon-1}{\varepsilon+2}a^3, \ee
%%%%%%%%%%%%%%%%%%%%%%%%%%%%%%%%%%%%%%%%%%%%%%%%%%%%%%%%%
where $\varepsilon$ denotes the permittivity of the object, $d$ the
size of the cubic meshsize and
$a=\left(\frac{3}{4\pi}\right)^{\frac{1}{3}}d$ the radius of the
sphere of the same volume than the cubic meshsize of the side
$d$. Unfortunately, this relation does not keep the energy and, then,
it is necessary to introduce a radiative reaction term that takes into
account the fact that charges in movement lose energy, and the
polarizability is, then, written as ~\cite{Draine_AJ_88}:
%%%%%%%%%%%%%%%%%%%%%%%%%%%%%%%%%%%%%%%%%%%%%%%%%%%%%%%%%
\be \alpha_{\rm RR} & = & \frac{\alpha_{\rm CM}}{1-\frac{2}{3} i k_0^3
  \alpha_{\rm CM}}. \ee
%%%%%%%%%%%%%%%%%%%%%%%%%%%%%%%%%%%%%%%%%%%%%%%%%%%%%%%%%
After different forms of the polarizability have been established in
order to improve the precision of the DDA and take into account the
non-punctual character of the dipole, and we may quote, among the best
known, the ones by Goedecke and O'Brien~\cite{Goedecke_AO_88},
%%%%%%%%%%%%%%%%%%%%%%%%%%%%%%%%%%%%%%%%%%%%%%%%%%%%%%%%%
\be \alpha_{\rm GB} & = & \frac{\alpha_{\rm CM}}{1-\frac{2}{3} i k_0^3
  \alpha_{\rm CM}-k_0^2 \alpha_{\rm CM}/a}, \ee
%%%%%%%%%%%%%%%%%%%%%%%%%%%%%%%%%%%%%%%%%%%%%%%%%%%%%%%%%
by Lakhtakia~\cite{Lakhtakia_IJMPC_92}:
%%%%%%%%%%%%%%%%%%%%%%%%%%%%%%%%%%%%%%%%%%%%%%%%%%%%%%%%%
\be \alpha_{\rm LA} & = & \frac{ \alpha_{\rm CM} }{1- 2
  \frac{\varepsilon-1}{\varepsilon+2}\left[ (1-i k_0 a)e^{i k_0
      a}-1\right] } \ee
%%%%%%%%%%%%%%%%%%%%%%%%%%%%%%%%%%%%%%%%%%%%%%%%%%%%%%%%%
and Draine and Goodman~\cite{Draine_AJ_93}
%%%%%%%%%%%%%%%%%%%%%%%%%%%%%%%%%%%%%%%%%%%%%%%%%%%%%%%%%
\be \alpha_{\rm LR} & = & \frac{ \alpha_{\rm CM}}{ 1 + \alpha_{\rm CM}
  \left[ \frac{(b_1+\varepsilon b_2 +\varepsilon b_3
      S)k_0^2}{d}-\frac{2}{3} i k_0^3 \right] }, \ee
%%%%%%%%%%%%%%%%%%%%%%%%%%%%%%%%%%%%%%%%%%%%%%%%%%%%%%%%%
with $b_1=-1.891531$, $b_2=0.1618469$, $b_3=-1.7700004$ and $S=1/5$.

Inside the code by default, it is $\alpha_{\rm RR}$ which is used. In
the case when the permittivity is anisotropic only $\alpha_{\rm RR}$
is going to be used.

A last polarizability is introduced (PS) that only works for
homogeneous spheres and is particularly precize for metals.  This
consists of making a change of the polarizability of the elements on
the edge of the sphere taking into account the factor of
depolarization of the sphere.~\cite{Rahmani_AJ_04}


\section{Correction to the tensor of susceptibility}

The tensor of susceptibility (or dyadic Green function) of the field
connects the dipole to the position $\ve{r}_j$ to the field radiated
at the position $\ve{r}_i$ by the relation :
$\ve{E}(\ve{r}_i)= \ve{T}(\ve{r}_i,\ve{r}_j)\ve{p}(\ve{r}_i)$. But
inside the DDA, considering the fact that the dipoles are associated
with a certain volume, the following integration should be
written~\cite{Chaumet_PRE_04}:
%%%%%%%%%%%%%%%%%%%%%%%%%%%%%%%%%%%%%%%%%%%%%%%%%%%%%%%%%
\be \ve{E}(\ve{r}_i)= \int_{V_j} \ve{T}(\ve{r}_i,\ve{r})\ve{p}(\ve{r})
{\rm d} \ve{r} \approx \left[ \int_{V_j} \ve{T}(\ve{r}_i,\ve{r}) {\rm
    d} \ve{r}\right] \ve{p}(\ve{r}_j), \ee
%%%%%%%%%%%%%%%%%%%%%%%%%%%%%%%%%%%%%%%%%%%%%%%%%%%%%%%%%
supposing the meshsize small enough to be able to consider the field
as being uniform in it.  So, the tensor must be integrated upon the
meshsize $V_j$. This integration is not analytic (it has to be done
numerically and this takes time) and, in fact, it only serves for the
dipoles which are the nearest to the observation, after that, the
integration does not bring any more precision. So, in the code, we
propose the possibility to integrate upon the nearest mesh sizes:
%%%%%%%%%%%%%%%%%%%%%%%%%%%%%%%%%%%%%%%%%%%%%%%%%%%%%%%%%
\be \int_{V_j} \ve{T}(\ve{r}_i,\ve{r}) {\rm d} \ve{r} & ~~~~~{\rm
  if}~~~~~& \frac{\|\ve{r}_i-\ve{r}_j\|}{d} \le n \\
\ve{T}(\ve{r}_i,\ve{r}_j) & ~~~~~{\rm if}~~~~~&
\frac{\|\ve{r}_i-\ve{r}_j\|}{d} \gt n. \ee
%%%%%%%%%%%%%%%%%%%%%%%%%%%%%%%%%%%%%%%%%%%%%%%%%%%%%%%%%
$n$ may take the value entire 0 (by default) until 5.

\section{Solve the system of linear equation}

In order to know the electric field in the object, {\it i.e.} the
field at the position of the $N$ elements of discretization, we have
to solve the following system of linear equation:
%%%%%%%%%%%%%%%%%%%%%%%%%%%%%%%%%%%%%%%%%%%%%%%%%%%%%%%%%
\be \ve{E} = \ve{E}_0 + \ve{A} D_\alpha \ve{E},\ee
%%%%%%%%%%%%%%%%%%%%%%%%%%%%%%%%%%%%%%%%%%%%%%%%%%%%%%%%%
where $\ve{E}_0$ is a vector of size $3N$ which contains the incident
field at the discretization elements. $\ve{A}$ is a matrix
$3N\times 3N$ which contains all the field tensor susceptibility and
$D_\alpha$ is a diagonal matrix $3N\times 3N$, if the object is
isotropic, or diagonal block $3\times 3$ if the object is anisotropic.
$\ve{E}$ is the vector $3N$ which contains the unknown electric local
fields. The equation is solved by a non-linear iterative method. The
code proposes numerous iterative methods, and the one used by default
is GPBICG because it is the most efficient in most cases
~\cite{Chaumet_OL_09}.  The code stops when the residue,
%%%%%%%%%%%%%%%%%%%%%%%%%%%%%%%%%%%%%%%%%%%%%%%%%%%%%%%%%%%%%%
\be r & = & \frac{ \|\ve{E}-\ve{A} D_\alpha \ve{E} -\ve{E}_0\|} {
  \|\ve{E}_0 \|}, \ee
%%%%%%%%%%%%%%%%%%%%%%%%%%%%%%%%%%%%%%%%%%%%%%%%%%%%%%%%%%%%%%
is under the tolerance given by the user. $10^{-4}$ is the tolerance
used by default, because it is a good compromise between speed and
precision. Please find below the different iterative method possible
in the code:
\begin{itemize}
\item GPBICG1 : Ref.~\onlinecite{Thuthu_IMECS_09}
\item GPBICG2 : Ref.~\onlinecite{Thuthu_IMECS_09}
\item GPBICGsafe : Ref.~\onlinecite{Fujino_IMECS_12}
\item GPBICGAR1 : Ref.~\onlinecite{Thuthu_IMECS_09}
\item GPBICGAR2 : Ref.~\onlinecite{Thuthu_IMECS_09}
\item QMRCLA : Ref.~\onlinecite{Cunha_ANM_95}
\item TFQMR : Ref.~\onlinecite{Cunha_ANM_95}
\item CG : Ref.~\onlinecite{Cunha_ANM_95}
\item BICGSTAB : Ref.~\onlinecite{Cunha_ANM_95}
\item QMRBICGSTAB1 : Ref.~\onlinecite{Chan_SIAMJSC_94}
\item QMRBICGSTAB2 : Ref.~\onlinecite{Chan_SIAMJSC_94}
\item GPBICOR : Ref.~\onlinecite{Zhao_CMA_13}
\item CORS : Ref.~\onlinecite{Carpentieri_CEI_CEIW}
\item BiCGstar-plus Ref.~\onlinecite{Fujino_WCE_15}
\end{itemize}
