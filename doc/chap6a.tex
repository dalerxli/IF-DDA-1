\chapter{Representation of the results}\label{chap6}
\markboth{\uppercase{Representation of the results}}{\uppercase{Representation of the results}}

\minitoc

\section{Introduction}

Three windows enable us to manage and represent the requested results.
The one on the top enables us to manage the different figures; the one
at the bottom on the left present the digital values requested, and
the one at the bottom on the right is kept for the graphic
representations.

\section{Digital exits}

All the results are given in the SI system.

\begin{itemize}

\item {\it Object subunits}: Number of elements of discretization of
  the  object under study.

\item {\it Mesh subunits} : Number of elements of discretization of
  the cuboid containing the object under study.

\item {\it Mesh size} : Size of the element of discretization.

\item $\lambda/(10|n|)$ : In order to obtain a good precision, it is
  advised to have a discretization under the value of $\lambda/10$ in
  the considered material of optical index $n$.

\item $k_0$ :Wave number.

\item {\it Irradiance}: Beam irradiance, for a Gaussian beam, it is
  estimated at the center of the waist.

\item {\it Field modulus}: Modulus of the field, for a Gaussian beam, it is 
estimated at the center of the waist.

\item {\it Tolerance obtained}: Tolerance obtained for the chosen iterative 
method. Logically under the requested value.

\item {\it Number of products Ax (iterations)}: Number of matrix
  vector products completed by the iterative method. Between brackets
  the iteration number of the iterative method.

\item {\it Extinction cross section}: Value of the extinction cross
  section.

\item {\it Absorbing cross section}: Value of the absorbing cross
  section.

\item {\it Scattering cross section}: Value of the scattering cross
  section obtained by = extinction cross section- absorbing cross
  section.

\item {\it Scattering cross section with integration}: Value of the
  scattering cross section obtained by integration of the far field
  field radiated by the object.

\item {\it Scattering asymmetric parameter}: Asymmetric factor.

\item {\it Optical force $x$}: Optical force according to the axis  $x$.

\item {\it Optical force $y$}: Optical force according to the axis  $y$.

\item {\it Optical force $z$}: Optical force according to the axis  $z$.

\item {\it Optical force modulus}: Modulus of the optical force.

\item {\it Optical torque $x$}:  Optical torque according to the axis  $x$.

\item {\it Optical  torque $y$}: Optical torque according to the axis  $x$.

\item {\it Optical  torque $z$}: Optical torque according to the axis  $x$.

\item {\it Optical torque modulus} Modulus of the optical torque.

\end{itemize}

\section{Graphics}

\subsection{Plot epsilon/dipoles}

The button {\it Plot epsilon/dipoles} enables us to see the position
of each element of discretization. The colour of each point is
associated with the value of the permittivity of the considered
meshsize.


\subsection{Far field and microscopy}

\subsubsection{Plot Poynting vector}

{\it Plot Poynting}: enables us to draw the modulus of the Poynting
vector in 3D.

\subsubsection{Plot microscopy}

{\it Plot microscopy} :enables us to draw the diffracted field in far
field by the object may this be either of the modulus of the field or
of the $x$, $y$ or $z$. Then, the vectorial field on the picture plane
is represented by considering a magnification $G$ for the microscope.

The diffracted field is represented upon a regular mesh in
$\Delta k_x=\Delta k_y$ such as $\sqrt{k_x^2+k_y^2} \le k_0$ NA with
the origin of the phase at the origin of the frame $(x,y,z)$. If the
computation is done by radiation of the dipoles, then, the obtained
picture has a size $k_0 NA$ and discretized as
$\Delta k_x=2 k_0 NA/N$, and if this one is done with Fourier
transform, then, the size of the picture is fixed by discretization of
the object $\Delta x$ with the relation $\Delta x \Delta k=2\pi/N$.

The field inside the picture plane is calculated with Fourier
transform.  So, we have with the calculation by radiation of the
dipoles:
%%%%%%%%%%%%%%%%%%%%%%%%%%%%%%%%%%%%%%%%%%%%%
\be
\Delta x \Delta k_x & = & \frac{2\pi}{N} \\
\Delta x 2 k_0{\rm NA} & = & 2\pi \\
\Delta x & = & \frac{\lambda}{2 {\rm NA}} \ee
%%%%%%%%%%%%%%%%%%%%%%%%%%%%%%%%%%%%%%%%%%%%%
The size of the picture is then $\lambda/(2 {\rm NA})$.

If the calculation of the diffracted field has been made by FFT, then,
the discretization is that of the mesh.

For the holographic microscope we present the Fourier plane (with and
without the incident field) and the image plane (with and without the
incident field). For the brightfield options we present the image
plane with the incident field (brightfield) and without the incident
field (a kind of darkfield). For the darkfield \& phase option we
present the image plane without the incident (darkfield) and with the
incident pahse shifted of $\pi/2$ (phase).


\subsection{Study of the near field}

\begin{itemize}

\item The first button {\it Field} enables us to choose to represent
  the incident field, local field or macroscopic field.

\item The button {\it Type} enables us to represent the modulus or the 
component $x$, $y$ or $z$ of the studied field.

\item The button {\it Cross section $x$} ($y$ or $z$) enables us to
  choose the abscissa of the cut (ordinate or dimension). {\it Plot
    $x$} ($y$ or $z$) draws the cut in plane $x$. {\it Plot all $x$}
  draws all the cut at once.

\end{itemize}

\subsection{optical force and torque}

\begin{itemize}

\item The first button {\it Field} enables us to choose to represent
  the optical force or the optical torque.

\item The button {\it Type} enables us to choose to represent the
  modulus or the component $x$, $y$ or $z$ of the studied field.

\item The button {\it Cross section $x$} ($y$ or $z$) enables us to
  choose the abscissa of the cut (ordinate or azimuth). {\it Plot $x$}
  ($y$ or $z$) draws the cur. {\it Plot all $x$} draws all the cuts at
  once.

\end{itemize}