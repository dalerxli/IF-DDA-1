\chapter{Definition of the object}\label{chap4}
\markboth{\uppercase{Definition of the object}}{\uppercase{Definition 
of the object}}

\minitoc

\section{Introduction}

The code proposes several predefined objects, and we are going to
precise in this section how to enter their optogeometrical
characteristics.  Note that all the distances have to be entered in
nanometers. The code is doing the conversion in meters.


\section{Type of the object}


The list of the predefined objects is the following: 

sphere, cube, cuboid, ellipsoid, several distinct spheres, cylinder,
concentric spheres, inhomogeneous sphere and arbitrary object.

When the objects as the cube or the cuboid have their edges turned
with respect to the axes of the system of coordinates, the angles of
Euler are used as defined in Fig.~\ref{euler}. The rotation centre
being the inertia centre of the object and the matrix of rotation
reads:
%%%%%%%%%%%%%%%%%%%%%
\be \ve{A} & = & \begin{pmatrix}\cos(\psi )\cos(\varphi )-\sin(\psi
  )\cos(\theta )\sin(\varphi )& -\cos(\psi )\sin(\varphi )-\sin(\psi
  )\cos(\theta )\cos(\varphi )& \sin(\psi )\sin(\theta )\\ \sin(\psi
  )\cos(\varphi )+\cos(\psi )\cos(\theta )\sin(\varphi )& -\sin(\psi
  )\sin(\varphi )+\cos(\psi )\cos(\theta )\cos(\varphi )& -\cos(\psi
  )\sin(\theta )\\ \sin(\theta )\sin(\varphi )& \sin(\theta
  )\cos(\varphi )& \cos(\theta )\end{pmatrix} \nn \ee
%%%%%%%%%%%%%%%%%%%%%%%%%%%%%%%%%%%%%%%%%%%%%%%
\begin{figure}[h]
\begin{center}
  \includegraphics*[width=8.0cm,draft=false]{euler.eps}
\end{center}
\caption{Definition of the angles of Euler according to the convention
  $Z-X-Z$. Scheme taken from Wikipedia}
\label{euler}
\end{figure}

%%%%%%%%%%%%%%%%%%%%%%%%%%%%%%%%%%%%%%%%%%%%%%%
\subsection{Sphere}

For the sphere, there are four fields to be filled:

\begin{itemize}
\item The radius of the sphere in nanometer
\item The abscissa of the centre of the sphere in nanometer
\item The ordinate of the centre of the sphere in nanometer
\item The azimuth of the centre of the sphere in nanometer
\end{itemize}

\subsection{Inhomogeneous sphere }


The permittivity of the sphere have a Gaussian noise with a
correlation length $l_c$, standard deviation, $A$ and an average
$\varepsilon_r$.

For the inhomogeneous sphere there are seven fields to be filled:

\begin{itemize}
\item The radius of the sphere in nanometer
\item The abscissa of the centre of the sphere in nanometer
\item The ordinate of the centre of the sphere in nanometer
\item The azimuth of the centre of the sphere in nanometer
\item The seed
\item The correlation length $l_c$
\item The magnitude of oscillation $A$
\end{itemize}


\subsection{Random sphere (length)}

All the spheres are constituted with the same permittivity and the
same radius but are distributed randomly in cuboid. There are nine
fields to be filled:

\begin{itemize}
\item The edge of the cube in nanometer according to the axis $x$
\item The edge of the cube in nanometer according to the axis $y$
\item The edge of the cube in nanometer according to the axis $z$
\item The abscissa of the centre of the cuboid in nanometer
\item The ordinate of the centre of the cuboid in nanometer
\item The azimuth of the centre of the sphere in nanometer
\item The seed
\item The radius of the spheres
\item The density of sphere, {\it i.e.}  $d=$volume of the sphere
  divided by the volume of the cuboid. $d$ should satisfy the inequality
  $0<d<0.5$. If $d$ is above 2, then it corresponds to the number of
  sphere in the box.
\end{itemize}

\subsection{Random sphere (meshsize)}

All the spheres are constituted with the same permittivity and the
same radius but are distributed randomly in cuboid. There are ten
fields to be filled:

\begin{itemize}
\item The abscissa of the centre of the cuboid in nanometer
\item The ordinate of the centre of the cuboid in nanometer
\item The azimuth of the centre of the sphere in nanometer
\item Number of meshsize long $x$
\item Number of meshsize long $y$
\item Number of meshsize long $z$
\item meshsize in nanometer
\item The radius of the spheres
\item The seed
\item The density of sphere, {\it i.e.}  $d=$volume of the sphere
  divided by the volume of the cuboid. $d$ should satisfy the inequality
  $0<d<0.5$. If $d$ is above 2, then it corresponds to the number of
  sphere in the box.
\end{itemize}

\subsection{Cube}

For the cube, there are seven fields to be filled:

\begin{itemize}
\item The edge of the cube in nanometer
\item The abscissa of the centre of the sphere in nanometer
\item The ordinate of the centre of the sphere in nanometer
\item The azimuth of the centre of the sphere in nanometer
\item First angle of Euler $\psi$ by rotation around the axis $z$
\item Second angle of Euler $\theta$ by rotation around the axis $x$
\item Third angle of Euler $\varphi$ by rotation around the axis $z$
\end{itemize}

\subsection{Cuboid (length)}

For the cuboid, there are nine fields to be filled:

\begin{itemize}
\item The edge of the cube in nanometer according to the axis $x$
\item The edge of the cube in nanometer according to the axis $y$
\item The edge of the cube in nanometer according to the axis $z$
\item The abscissa of the centre of the cuboid in nanometer
\item The ordinate of the centre of the cuboid in nanometer
\item The azimuth of the centre of the sphere in nanometer
\item First angle of Euler $\psi$ by rotation around the axis $z$
\item Second angle of Euler $\theta$ by rotation around the axis $x$
\item Third angle of Euler $\varphi$ by rotation around the axis $z$
\end{itemize}

\subsection{Cuboid (meshsize)}

For the cuboid, there are seven fields to be filled:

\begin{itemize}
\item The abscissa of the centre of the cuboid in nanometer
\item The ordinate of the centre of the cuboid in nanometer
\item The azimuth of the centre of the sphere in nanometer
\item Number of meshsize long $x$
\item Number of meshsize long $y$
\item Number of meshsize long $z$
\item Meshsize in nanometer
\end{itemize}


\subsection{Inhomogeneous Cuboid (length)}


The permittivity of the cuboid have a Gaussian noise with a
correlation length $l_c$, standard devaition $A$ and an average
$\varepsilon_r$.  For the cuboid, there are nine fields to be filled:

\begin{itemize}
\item The edge of the cube in nanometer according to the axis $x$
\item The edge of the cube in nanometer according to the axis $y$
\item The edge of the cube in nanometer according to the axis $z$
\item The abscissa of the centre of the cuboid in nanometer
\item The ordinate of the centre of the cuboid in nanometer
\item The azimuth of the centre of the sphere in nanometer
\item The seed
\item The correlation length $l_c$
\item The magnitude of oscillation $A$

\end{itemize}


\subsection{Inhomogeneous Cuboid (meshsize)}


The permittivity of the cuboid have a Gaussian noise with a
correlation length $l_c$, standard devaition $A$ and an average
$\varepsilon_r$.  For the cuboid, there are nine fields to be filled:

\begin{itemize}
\item The abscissa of the centre of the cuboid in nanometer
\item The ordinate of the centre of the cuboid in nanometer
\item The azimuth of the centre of the sphere in nanometer
\item Number of meshsize long $x$
\item Number of meshsize long $y$
\item Number of meshsize long $z$
\item Meshsize in nanometer
\item The seed
\item The correlation length $l_c$
\item The magnitude of oscillation $A$

\end{itemize}


\subsection{Ellipsoid}

For the ellipsoid, there are nine fields to be fulfilled:

\begin{itemize}
\item The half axis in nanometer according to the axis $x$
\item The half axis in nanometer according to the axis $y$
\item The half axis in nanometer according to the axis $z$
\item The abscissa of the centre of the ellipse in nanometer
\item The ordinate of the centre of the ellipse in nanometer
\item The azimuth of the centre of the ellipse in nanometer
\item First angle of Euler $\psi$ by rotation around the axis $z$
\item Second angle of Euler $\theta$ by rotation around the axis $x$
\item Third angle of Euler $\varphi$ by rotation around the axis $z$
\end{itemize}


\subsection{Several spheres}

For multiple spheres, it is convenient first to choose with the line 
from the under {\it number of objects} the number $N$ of the expected
spheres. Then, when we click on {\it Props} $N$ windows, that we fill 
in the same way as for the unique sphere, appear. 
Beware, the spheres must be disconnected, otherwise, the code stops 
and shows error. 

\subsection{Cylinder}

For the cylinder, there are eight fields to be fulfilled:

\begin{itemize}
\item The radius of the cylinder in nanometers
\item The length of the cylinder in nanometer
\item The abscissa of the centre of the cylinder in nanometer
\item The ordinate of the centre of the cylinder in nanometer
\item The azimuth of the centre of the cylinder in nanometer
\item First angle of Euler $\psi$ by rotation around the axis $z$
\item Second angle of Euler $\theta$ by rotation around the axis $x$
\item Third angle of Euler $\varphi$ by rotation around the axis $z$
\end{itemize}


\subsection{concentric spheres}

For concentric spheres, it is convenient first to choose with the
under line {\it number of objects} the number $N$ of concentric
spheres. Then, when we click on {\it Props} $N$ windows appear. The
first window is filled the same way as for the sphere, and for the
next windows, it is enough to enter the radius in nanometer.  The
radii must be entered in increasing order, otherwise, the code
shows the error.

\subsection{Arbitrary object}

In the case of an arbitrary object, it is defined by the user. In
other words, he has to create the object himself, and then, it is
convenient to create this entry file by respecting the conventions
chosen by the code. {\it namefile} is the name of the file containing
the arbitrary object and it is asked for when we choose the arbitrary
object. It is coded in sequential and in ascii, and is necessarily
described inside a cuboid box. Below are given the lines of
the code enabling to create this file:

\hspace{5mm} open(15,file=namefile,status='old',iostat=ierror)

\hspace{5mm} write(15,*) nx,ny,nz

\hspace{5mm} write(15,*) aretecube


\hspace{5mm} {\bf do} i=1,nz

  \hspace{10mm} {\bf do} j=1,ny

 \hspace{15mm} {\bf do} k=1,nx
  

      \hspace{20mm}  write(15,*) xs(i,j,k),ys(i,j,k),zs(i,j,k)     
         
   \hspace{15mm} {\bf enddo}

  \hspace{10mm} {\bf enddo}

 \hspace{5mm} {\bf enddo}

\hspace{5mm} {\bf do} i=1,nz

  \hspace{10mm} {\bf do} j=1,ny

 \hspace{15mm} {\bf do} k=1,nx

 \hspace{20mm} {\bf if}  objet isotrope 

         \hspace{25mm}    write(15,*)  eps(i,j,k)
                    
    \hspace{20mm} {\bf elseif}  objet anisotrope

       \hspace{25mm}  {\bf do} ii=1,3
 
              \hspace{30mm}  {\bf do} jj=1,3

                \hspace{35mm} write(15,*) epsani(ii,jj,i,j,k)

   \hspace{30mm} {\bf enddo}

  \hspace{25mm} {\bf enddo}

       \hspace{20mm} {\bf endif}

    \hspace{15mm} {\bf enddo}

  \hspace{10mm} {\bf enddo}

 \hspace{5mm} {\bf enddo}


\vspace{10mm}

\begin{itemize}
\item nx : size of the cuboid according to the axis $x$.
\item ny : size of the cuboid according to the axis $y$.
\item nz : size of the cuboid according to the axis $z$.
\item aretecube : size of the meshsize of discretization. 
\item x : abscissa of the mesh of discretization according the axis
  $x$.
\item y : ordinate of the mesh of discretization according the axis
  $y$.
\item z : azimuth of the mesh of discretization according the axis
  $z$.
\item eps : epsilon of the object if isotropic
\item epsani : epsilon of the object if anisotropic
\end{itemize}

\section{Choose the relative permittivity}

When the object or objects are chosen, it is then convenient to enter
the relative permittivity. Apart from the arbitrary object, all the
defined objects by default in the code are considered as being
homogeneous. They may be isotropic or anisotropic. So, we choose {\it
  iso} or {\it aniso} and we click on {\it Epsilon}.

\begin{itemize} 
\item {\it iso}: A board appears, where either we enter the relative
  permittivity by hand (real and imaginary part) or we choose a
  material in the data base.
\item {\it aniso}: A board appears where we enter the relative
  permittivity by hand (real and imaginary part) for all the
  components of anisotropic tensor.
\end{itemize}

\section{Choose the discretization}

The number $N_c$ entered in the field of the discretization
corresponds to the number of layers forming the object in its largest
direction.

A few examples:

\begin{itemize} 
\item For an ellipse of half axis $(a,b,c)$, it is going to be the
  greatest half axis $a$ that is going to be selected and the edge of
  discretization is going to be of $2a/N_c$.
\item For a cube the number of meshsize is so going to be of
  $N=N_c^3$.
\end{itemize}
