\chapter{Properties of the illumination }\label{chap3}
\markboth{\uppercase{Properties of the illumination}}{\uppercase
{Properties of the illumination}}

\minitoc

\section{Introduction}

In the section properties of the illumination, the field {\it
  Wavelength} enables us to enter the using wavelength. This one is
entered in nanometer. The field $P_0$ enables to enter the power of
the laser beam in Watt. The field $W_0$ in nanometer enables to enter
for a plane wave the radius of the laser beam and for a Gaussian beam,
the waist of the beam.

\section{Beam}

\subsection{Introduction}

There are six beams predefined, their propagation direction is always
defined in the same way, with two angles $\theta$ and $\varphi$.  They
are connected to the given direction by the wave vector as follows:
%%%%%%%%%%%%%%%%%%%%%%%%%%%%%%%%%%%%%%%%%%%%%
\be k_x & = & k_0 \sin \theta \cos\varphi \\
k_y & = & k_0 \sin \theta \sin\varphi \\
k_z & = & k_0 \cos \theta \ee
%%%%%%%%%%%%%%%%%%%%%%%%%%%%%%%%%%%%%%%%%%%%%
where $\ve{k}_0=(k_x,k_y,k_z)$ is the wave vector parallel to the direction
of the incident beam and $k_0$ the wave number, see
Fig.~\ref{faisceau}.
%%%%%%%%%%%%%%%%%%%%%%%%%%%%%%%%%%%%%%%%%%%%%%%
\begin{figure}[h]
\begin{center}
  \includegraphics*[width=8.0cm,draft=false]{faisceau.eps}
\end{center}
\caption{Definition of the beam's direction}
\label{faisceau}
\end{figure}
%%%%%%%%%%%%%%%%%%%%%%%%%%%%%%%%%%%%%%%%%%%%%%%
For the polarization, we use the plane $(x,y)$ as referential surface.
Then, we can determine a polarization TM ($p$) and TE ($s$) with the
presence of a surface, see Fig.~\ref{pola}.
%%%%%%%%%%%%%%%%%%%%%%%%%%%%%%%%%%%%%%%%%%%%%%%
\begin{figure}[h]
\begin{center}
  \includegraphics*[width=8.0cm,draft=false]{pola.eps}
\end{center}
\caption{Definition of the beam's polarization.}
\label{pola}
\end{figure}
%%%%%%%%%%%%%%%%%%%%%%%%%%%%%%%%%%%%%%%%%%%%%%%
The frame $(x,y,z)$ is used as an absolute referential.

\subsection{Linear plane wave }

{\it Linear plane wave} is a plane wave linearly polarized. The first line
is relative to $\theta$ and the second to $\varphi$. The third line is
connected to the polarization, pola=1 en TM and pola=0 in TE.  Note
that the polarization is not necessarily purely in TE or TM:
${\rm pola}\in[0~1]$ such as $E^2_{\rm TM}={\rm pola}^2E^2$ and
$E^2_{\rm TE}=(1-{\rm pola}^2)E^2$.

Note that the phase is always taken null at the origin of the frame:
%%%%%%%%%%%%%%%%%%%%%%%%%%%%%
\be \ve{E}(\ve{r})= \ve{E}_0 e^{i\ve{k}.\ve{r}}, \ee
%%%%%%%%%%%%%%%%%%%%%%%%%%%%%
with ${\rm Irradiance}=P_0/S$ where $S=\pi w_0^2$ is the surface of
the beam and $E_0=\sqrt{2 {\rm Irradiance}/c/\varepsilon_0}$.


\subsection{Circular plane wave }

{\it pwavecircular } is a plane wave circularly polarized. The first
line is relative to $\theta$ and the second to $\varphi$. The third
line is connected to the polarization that we can choose right (1) or
left (-1) circular.

Note that the phase is taken null at the origin of the frame.
%%%%%%%%%%%%%%%%%%%%%%%%%%%%%
\be \ve{E}(\ve{r})= \ve{E}_0 e^{i\ve{k}.\ve{r}}, \ee
%%%%%%%%%%%%%%%%%%%%%%%%%%%%%
with ${\rm Irradiance}=P_0/S$ where $S=\pi w_0^2$ is the surface of
the beam and $E_0=\sqrt{2 {\rm Irradiance}/c/\varepsilon_0}$.


\subsection{Multiplane wave}

{\it Multiplane wave} consists to take many planes waves. The first
thing to do is to choose the number of plane wave, and then for each
plane wave we choose $\theta$ and $\varphi$ and the polarization. We
have to write also the complex magnitude of each plane wave. The sum
of the power of all the plane wave is equal to $P_0$.


\subsection{Antenna}

The incident beam can be a dipole where the user defines the position
and orientation. Notice that the antenna can be inside or outside the
object. The magnitude is chosen such that the power radiated by the
dipole is equal to $P_0$:
%%%%%%%%%%%%%%%%%%%%%%%%%%%%%%
\be P_0 & = & \frac{1}{4\pi\varepsilon_0} \frac{k^4 c}{3} \| \ve{p}
\|^2 . \ee
%%%%%%%%%%%%%%%%%%%%%%%%%%%%%%

\subsection{Linear Gaussian}

{\it Linear Gaussian} is a Gaussian wave polarized linearly. The first
line is relative to $\theta$ and the second to $\varphi$. The third
line is connected to the polarization pola=1 in TM and pola=0 in TE.
Note that the polarization is not necessarily in TE or TM:
${\rm pola}\in[0~1]$ such as $E^2_{\rm TM}={\rm pola}^2E^2$ and
$E^2_{\rm TE}=(1-{\rm pola}^2)E^2$.

The three following lines help to fix the position of the centre of
the waist in nanometers in the frame $(x,y,z)$.

Note that this Gaussian beam may have a very weak waist, because it is
calculated without any approximation through an angular spectrum
representation.  The definition of the waist, for a beam propagating
along the $z$ axis is :\cite{Agrawal_JOSA_79}
%%%%%%%%%%%%%%%%%%%%%%%%%%%%%
\be E(x,y,0)= E_0 e^{-\rho^2/(2 w_0^2)}, \ee
%%%%%%%%%%%%%%%%%%%%%%%%%%%%%
with $\rho=\sqrt{x^2+y^2}$. From this definition of the beam at $z=0$,
for a beam polarized along the $x$ axis we get
:\cite{Chaumet_JOSAA_06}
%%%%%%%%%%%%%%%%%%%%%%%%%%%%%
\be E_x & = & E_0 \int_{0}^{k_0} w_0^2
\exp\left(-\frac{w_0^2(k_0^2-k_z^2)}{2}\right) \exp(ik_z z)
J_0\left(\rho\sqrt{k_0^2-k_z^2}\right) k_z {\rm d}k_z \\ E_z & = & -i
E_0 \frac{x}{\rho} \int_{0}^{k_0} w_0^2
\exp\left(-\frac{w_0^2(k_0^2-k_z^2)}{2}\right) \exp(ik_z z)
J_1\left(\rho\sqrt{k_0^2-k_z^2}\right)\sqrt{k_0^2-k_z^2} {\rm d}k_z,
\ee
%%%%%%%%%%%%%%%%%%%%%%%%%%%%%
with $J_1$ and $J_0$ the Bessel's function. The irradiance is computed
at the center of the Gaussian beam and the relationship between the
power and the magnitude $E_0$ is:
%%%%%%%%%%%%%%%%%%%%%%%%%%%%%%%%%
\be P_0 & = & \frac{\pi w_0^2}{ 4 } c\varepsilon_0 E_0^2\left(
  1+\frac{(k_0 w_0)^2-1}{k_0 w_0} \frac{\sqrt{\pi}}{2} {\rm Im}[
  w(k_0w_0)] \right) \\
{\rm Irradiance} & = & \frac{E_0^2}{ 4 } c\varepsilon_0 \left(
  1+\frac{(k_0 w_0)^2-1}{k_0 w_0} \frac{\sqrt{\pi}}{2} {\rm Im}\left[
    w(k_0w_0/\sqrt{2})\right]  \right) , \ee
%%%%%%%%%%%%%%%%%%%%%%%%%%%%%%
where $w()$ denotes the Faddeeva's function. If we suppose
$w()\approx 0$, we obtain $P_0=\pi w_0^2 {\rm Irradiance}$ and we find
the relation given for a plane wave.


\subsection{Circular Gaussian}

{\it Circular Gaussian} is a Gaussian wave circularly polarized. The
first line is relative to $\theta$ and the second to $\varphi$. The
third line is connected to the polarization that we can choose right
(1) or left (-1) circular.


The next three lines enable us to fix the position of the centre of the waist
in nanometers in the frame $(x,y,z)$.

Note that this Gaussian wave may have a very weak waist, because it is 
calculated without any approximation through a plane wave spectrum.

\subsection{Circular and linear Gaussian (FFT)}

{\it Circular and linear Gaussian (FFT)} is a Gaussian wave based on
the previous computation for the Circular and linear Gaussian,
respectively. In this case, the incident wave is computed at the
bottom of the object and then the beam is propaged with FFT as for the
beam propagation method. This computation is quicker than the
rigourous one. However, one needs to choose the number of points for
the FFT enough large to not troncate the Gaussian beam and avoid
periodicity problem.

\subsection{Linear Guaussian (para)}


{\it Linear Guaussian (para)} is a Gaussian wave polarized
linearly. The first line is relative to $\theta$ and the second to
$\varphi$. The third line is connected to the polarization, pola=1
with TM and pola=0 with TE.  Note that the polarization is not
necessarily purely in TE or TM: ${\rm pola}\in[0~1]$ such as
$E^2_{\rm TM}={\rm pola}^2E^2$ and $E^2_{\rm TE}=(1-{\rm pola}^2)E^2$.

The next three lines enables us to fix the position of the centre of
the waist in nanometers in the frame $(x,y,z)$.

Note that this Gaussian wave is calculated in accordance with the
paraxial approximation and as such does not satisfy rigorously the
Maxwell's equations. For a wave propagating along the $z$ direction
and polarized along the $x$ axis we have:
%%%%%%%%%%%%%%%%%%%%%%%%%%%%%%%%%%%%
\be E_x & = & E_0 \sqrt{2}\frac{w_0}{w}e^{-\rho^2/w^2}e^{i k_0 \rho^2
  R(z)/2 }e^{i(k_0 z+\eta)} \\
w & = & \sqrt{2} w_0 \sqrt{1+\frac{z^2}{z_0^2} } \\
z_0 & = & k_0 w_0^2 \\
R(z) & = & \frac{z}{z^2+z_0^2} \\
\eta & = & \tan^{-1}(z/z_0). \ee
%%%%%%%%%%%%%%%%%%%%%%%%%%%%%%%%%%%%
We remark that for $z=0$ the Gaussian beam has the same magnitude that
those computed rigorously. The field and the irradiance at the center
of the waist are computed through
%%%%%%%%%%%%%%%%%%%%%%%%%%%%%%%%%%%%
\be E_0 & = & \sqrt{\frac{2 P_0}{\pi c \varepsilon_0 w_0^2} } \\
{\rm irradiance} & = & c \varepsilon_0 E_0^2/2 = \frac{P_0}{\pi w_0^2}
.\ee
%%%%%%%%%%%%%%%%%%%%%%%%%%%%%%%%%%%%


\subsection{Circular Gaussian (para)}

{\it Circular Gaussian (para)} is a Gaussian wave polarized
circularly.  The first line is relative to $\theta$ and the second to
$\varphi$.  The third line is connected to the polarization that we
may choose right or left.


The next three lines enable us to fix the position of the centre of
the waist in nanometers in the frame $(x,y,z)$.

Note that this Gaussian wave is calculated in accordance with the
paraxial approximation and as such does not satisfy rigorously the
Maxwell's equations.

\subsection{Arbitrary wave}

In the case of an arbitrary field, the characteristic are determined
by the user.  In other words, he has to create the field himself, and
it is mandatory to create these files respecting the chosen
conventions by the code.


The description of the discretization of the incident field is done 
within a file which is asked for when we click on {\it Props}.
For example, for the real part of the component $x$ of the field, 
it has to be constructed as follows:

nx,ny,nz 

dx,dy,dz

xmin,ymin,zmin

\begin{itemize}
\item  nx is the number of meshsize according to the axis $x$
\item  ny is the number of meshsize according to the axis $y$
\item  nz is the number of meshsize according to the axis $z$
\item  dx is the step according to the axis $x$
\item  dy is the step according to the axis $y$
\item  dz is the step according to the axis $z$
\item xmin the smallest abscissa
\item ymin the smallest ordinate
\item zmin the smallest azimuth
\end{itemize}

Then, the files of the electric field are created as follows for 
each of the components of the real part and separated imaginary field:

\vspace{10mm}

open(11, file='Exr.mat', status='new', form='formatted', access='direct', recl=22)

{\bf do} k=1,nz

\hspace{5mm} {\bf do} j=1,ny

\hspace{10mm} {\bf do} i=1,nx 

\hspace{15mm} ii=i+nx*(j-1)+nx*ny*(k-1)

\hspace{15mm} write(11,FMT='(D22.15)',rec=ii) dreal(Ex)

\hspace{10mm} {\bf enddo}

\hspace{5mm} {\bf enddo}

{\bf enddo}

\vspace{10mm}

Be careful, the mesh size of the discretization of the object has to
be larger than the meshsize of the discretization of the field.
