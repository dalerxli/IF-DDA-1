\chapter{Possible study with the code}\label{chap5}
\markboth{\uppercase{Possible study with the
    code}}{\uppercase{Possible study with the code}}

\minitoc

\section{Introduction}

To determine the object with the appropriate orientation is not an
easy task.  That is why the first option {\it Only dipoles with
  epsilon}, enables us to check quickly if the object entered is well
the one intended without any calculation being launched. Once this has
been done, there are three great fields: the study in far field, the
study in near field and the optical forces.

\vskip10mm

{\underline{Important}}: Note that in the DDA the computation that
takes the longest time is the calculation of the local field due to
the necessity to solve the system of linear equations.  One option has
been added which consists in reading again the local field starting
with a file. When this option is selected, the name of a file is asked
for; either we enter an old file or a new name:

\begin{itemize}
\item If this is a new name, the calculation of the local field is
  going to be accomplished, then, stored together with the chosen
  configuration.
\item - If this is an old name, the local field is going to be read
  again with a checking that the configuration has not been changed
  between the writing and the second reading. This makes it easier to
  relaunch calculations very quickly for the same configuration but
  for different studies.
\end{itemize}


Note also that if the calculation asked has a large number of
discretization and that we are not interested by the output files in
.mat (needs to use matlab), then we have the option ``Do not write mat
file''. This requires the code to write no .mat file, and allows the
code to go faster, less fill the hard drive and be better
parallelized.


\section{Study in far field}

When the option far field is selected, three possibilities appear:

\begin{itemize}

\item {\it Cross section}: This option enables us to calculate the
  extinction ($C_{\rm ext}$), absorbing ($C_{\rm abs}$) and scattering
  cross section ($C_{\rm sca}$). The scattering cross section is
  obtained through $C_{\rm sca}=C_{\rm ext}-C_{\rm abs}$. The
  extinction and absorption cross sections may be evaluated as:
%%%%%%%%%%%%%%%%%%%%%%%%%%%%%%%%%%%%%%%%%%%%%%%%
  \be C_{\rm ext} & = & \frac{4\pi k_0}{\|\ve{E}_0\|^2} \sum_{j=1}^{N}
  {\rm Im} \left[ \ve{E}^*_0(\ve{r}_j).  \ve{p}(\ve{r}_j) \right] \\
  C_{\rm abs} & = & \frac{4\pi k_0}{\|\ve{E}_0\|^2} \sum_{j=1}^{N}
  \left[ {\rm Im} \left[ \ve{p}(\ve{r}_j). (\alpha^{-1}(\ve{r}_j))^*
      \ve{p}^*(\ve{r}_j) \right] -\frac{2}{3} k_0^3
    \| \ve{p}^*(\ve{r}_j) \|^2 \right] \ee
%%%%%%%%%%%%%%%%%%%%%%%%%%%%%%%%%%%%%%%%%%%%%%%%

\item {\it Cross section+Poynting}: This option calculates also the
  scattering cross section from the integration of the far field
  diffracted by the object upon 4$\pi$ st�radians, the asymmetric
  factor and calculates differential cross section, {\it i.e.}
  $\left< \ve{S} \right> .\ve{n} R^2$ with $\ve{S}$ the Poynting
  vector, $\ve{n}$ the direction of observation, which is going to be
  represented in 3D. The values {\it Ntheta} and {\it Nphi} enable us
  to give the number of points used in order to calculate the
  scattering cross and to represent the Poynting vector. The larger
  the object is, the larger {\it Ntheta} and {\it Nphi} must be, which
  leads to time consuming calculations for objects of several
  wavelengths.
%%%%%%%%%%%%%%%%%%%%%%%%%%%%%%%%%%%%%%%%%%%%%%%%
  \be C_{\rm sca} & = & \frac{k_0^4}{\|\ve{E}_0\|^2} \int \left\|
    \sum_{j=1}^N \left[ \ve{p}(\ve{r}_j)-\ve{n}(\ve{n}.
      \ve{p}(\ve{r}_j)) \right] e^{-i k_0 \ve{n}.\ve{r}_j} \right\|^2
  {\rm d}\Omega \\ g & = & \frac{k_0^3}{C_{\rm sca} \|\ve{E}_0\|^2}
  \int \ve{n}.\ve{k}_0 \left\| \sum_{j=1}^N \left[
      \ve{p}(\ve{r}_j)-\ve{n}(\ve{n}.  \ve{p}(\ve{r}_j)) \right] e^{-i
      k_0 \ve{n}.\ve{r}_j} \right\|^2 {\rm d}\Omega \\
  \left< \ve{S} \right> .\ve{n} R^2 & = & \frac{c k_0^4}{8\pi }
  \left\| \sum_{j=1}^N \left[ \ve{p}(\ve{r}_j)-\ve{n}(\ve{n}.
      \ve{p}(\ve{r}_j)) \right] e^{-i k_0 \ve{n}.\ve{r}_j} \right\|^2
  \ee
%%%%%%%%%%%%%%%%%%%%%%%%%%%%%%%%%%%%%%%%%%%%%%%%
  

  Another solution in order to go faster (option {\it quick
    computation}) and to pass by FFT for the calculation of the
  diffracted field.  In this case, of course, it is convenient to
  discretize keeping in mind that the relation
  $\Delta x \Delta k=2\pi/N$ connects the mesh size of the
  discretization with the size of the FFT. The $N$ chosen for the
  moment is $N=256$. This is convenient for objects larger than the
  wavelength. Indeed, $L=N\Delta x$ corresponds to the size of the
  object which gives $\Delta k=2\pi/L$, and if the size of the object
  is too small, then, the $\Delta k$ is too large, and the quadrature
  is imprecise. Note that since the integration is performed on two
  planes parallel to the plane $(x,y)$, is not convenient if the
  incident makes an angle more than 70 degrees with the $z$ axis. The
  3D representation of the vector of Poynting is done as previously,
  i.e. with {\it Ntheta} and {\it Nphi} starting with an interpolation
  upon the calculated points with the FFT.

\item {\it Energy Conservation}. This study computes the reflectance,
  transmittance and absorptance. If the object under study is no
  absorbing then the absorptance should be zero. Then it traduces the
  level of energy conservation of our solver. It can depend of the
  precision of the iterative method and of the polarizability chosen.

\end{itemize}
  
\section{Microscopy}
  
This option permits to compute the image obtained for different
microscope (holographic, brightfield, darkfield and phase). It asks
for the numerical aperture of the objective lens (necessarily between
0 and 1), then, calculates the field diffracted by the object and the
picture obtained through the microscope. By default, the lenses are
placed parallel to the plane $(x,y)$ and at the side of the positive
$z$. The focus of the microscope is placed to the origin of the frame
but can be chaned via the field ``Position of the focal plane''.
(Fig.~\ref{lentille}). The magnification of the microscope is $G$ and
should be above 1.

\begin{figure}[h]
\begin{center}
\includegraphics*[draft=false,width=150mm]{lentille.eps}
\caption{Simplified figure of the microscope. The object focus of the
  objective lens is at the origin of the frame. The axis of the lens
  is confounded with the $z$ axis and at the side of the positive
  $z$.}
\label{lentille}
\end{center}
\end{figure}


The calculation for the diffracted field may be completed starting
with the sum of the radiation of the dipoles (very long when the
object has a lot of dipoles) or with FFT (option {\it quick
  computation}) with a value $N=128$ by default here as well. In this
case, $\Delta x \Delta k=2\pi/N$ with $\Delta x$ the mesh size of
discretization of the object which corresponds also to the
discretization of the picture plane. Consequently, this one has a size
of $L=N \Delta x$.

The diffracted field in far field at a distance $r$ of the origin can
written as
$\ve{E}= \ve{S}(k_x,k_y,\ve{r}_{\rm object}) \frac{e^{i k r}}{r}$. The
field after the first lens is then defined as:
$\ve{E}^f=\frac{\ve{S}(k_x,k_y,\ve{r}_{\rm object})}{-2 i \pi \gamma}$
with $\gamma=\sqrt{k_0^2-k_x^2-k_y^2}$ and the image through the
microscope is given by its Fourier transform,
$\ve{E}^i= {\cal F}(\ve{E}^f)$.

To take into account the magnification of the microscope in the image
we perform a rotation of the vector $\ve{E}^f$ before its Fourier
transform as:
%%%%%%%%%%%%%%%%%%%%%%%%%%%%%%%%%%%%%%%%%%
\be\ve{E}^i & = & {\cal F}(R(\theta) \ve{E}^f) \\
{\rm with~} R(\theta) & = & \left( \begin{matrix} u_x^2
    +\cos\theta(1-u_x^2) & u_x u_y (1-\cos\theta) & u_y \sin\theta \\
    u_x u_y (1-\cos\theta) & u_y^2 +\cos\theta(1-u_y^2) & -u_x
    \sin\theta \\ -u_y \sin\theta
    & u_x \sin\theta & \cos\theta  \end{matrix} \right) \\
\theta & = & \sin^{-1} [  \sin(-\beta)/G] - \beta \\
\beta & = & \cos^{-1}(k_z/k_0) \\
u_x & = & -k_y/k_{\parallel}\\
u_y & = & k_x/k_{\parallel} . \ee
%%%%%%%%%%%%%%%%%%%%%%%%%%%%%%%%%%%%%%%%
Note that we can simulate a microscope in tansmission $(k_z>0)$ or un
reflexion $(k_z<0)$.  We can notice, for a microscope in transmission,
that when the total field is computed in the Fourier plane (scattered
plus incident filed {\it i.e.} specular), in the case of the plane
wave, a Dirac in the Fourier space is placed at the pixel the closest
of the incident wave vector.


\begin{itemize}

\item {\it Holographic}: This option computes the diffracted field
  (Fourier plane) with the incident field defined in the section
  illumination properties. It computes the image plane with or without
  the presence of the incident field.


  
\item {\it Brightfield}: This microscope uses a condenser lens, which
  focuses light from the light source onto the sample with a numerical
  aperture defined below the magnification. It consists to sum many
  incident field inside this numerical aperture with different
  polarizations, hence it can take time as it needs to solve many
  direct problem. The result is given in the image plane with the
  incident field (a kind of dark field) and with the incident field
  (brightfield).

\item {\it Darkfield \& phase}: In darkfield microscopy the condenser
  is designed to form a hollow cone of light with a numerical aperture
  equal to the condenser lens, as apposed to brightfield microscopy
  that illuminates the sample with a full cone of light. The result is
  given in the image plane (scattered field). In the phase microscopy
  the ring-shaped illuminating light that passes the condenser annulus
  is focused on the specimen by the condenser exactly as in the dark
  field microscope and then the incident field with a phase shifted of
  $\pi/2$ is added to the scattered field.
  
\end{itemize}

\section{Study in near field}

When the option near field is selected, two possibilities appear:

\begin{itemize}

\item {\it Local field}: This option enables us to draw the local
  field to the position of each element of discretization. The local
  field being the field at the position of each element of
  discretization in absence of itself. 

\item {\it Macroscopic field}: This option enables us to draw the
  macroscopic field to the position of each element of
  discretization. The connection between the local field and the
  macroscopic field is given Ref.~\cite{Chaumet_PRE_04} :
%%%%%%%%%%%%%%%%%%%%%%%%%%%%%%
  \be \ve{E}_{\rm macro} & = & 3 \left( \varepsilon+2 -i \frac{k_0^3
      d^3 }{2 \pi} (\varepsilon-1)\right)^{-1} \ve{E}_{\rm local} \ee
%%%%%%%%%%%%%%%%%%%%%%%%%%%%%%


\end{itemize}

The last option enables us to choose the mesh in which the local and
macroscopic fields are represented.

\begin{itemize}

\item {\it Object}: Only the field in the object is
  represented. Notice that when FFT is used for the beam or for the
  computation of the diffracted field then this options is passed in
  the option {\it Cube}. This is same for the computation of the
  emissivity, teh reread option and the use of the BPM(R).

\item {\it Cube}: The field is represented within a cube containing
  the object.

\item {\it Wide field}: The field is represented within a box greater
  than the object.  The size of the box correspond to the size of the
  object plus the Additional sideband ($x$, $y$ ou $z$) on each
  side. For example for a sphere with a radius $r=100$~nm and
  discretization of 10, {\it i.e.} a meshsize of 10 nm, with an
  Additional sideband $x$ of 2, 3 for $y$ and 4 for $z$, we get a box
  of size:
%%%%%%%%%%%%%%%%%%%%%%%%%%%%%%%%%%%%%%%%%%
  \be l_x & = & 100 + 2\times 2 \times 10 = 140~{\rm nm} \\
  l_y & = & 100 + 2\times 3 \times 10 = 160~{\rm nm} \\
  l_z & = & 100 + 2\times 4 \times 10 = 180~{\rm nm} \\
  \ee
\end{itemize}

\section{Optical force and torque}

When the force option is selected, four possibilities appear:
\begin{itemize}

\item {\it Optical force}: Calculation of the optical force exerting
  on one or more objects.

\item {\it Optical force density}: Enables us to draw the density of
  the optical force.

\item {\it Optical torque}: Calculation of the optical torque exerting
  on one or more objects.  The torque is computed for an origin placed
  in the gravity center of the object.

\item {\it Optical torque density}: Enables us to draw the density of
  the optical force torque.
\end{itemize}
The net optical force and troque experienced by the object are
computed with~\cite{Chaumet_OL_00,Chaumet_JAP_07a}:
%%%%%%%%%%%%%%%%%%%%%%%%%%%%%%%%%%%%%%%%%%%%%%%%
\be \ve{F} & = & (1/2) \sum_{j=1}^N {\rm Re}\left(\sum_{v=1}^{3}
  p_v(\ve{r}_j) \frac{\partial (E_v(\ve{r}_j))^*}{\partial u}\right) \\
\ve{\Gamma} & = & \sum_{j=1}^N \left[ \ve{r}_{j} \times
  \ve{F}(\ve{r}^g_{j})+ \frac{1}{2} {\rm Re} \left\{ \ve{p}(\ve{r}_{j})
    \times \left[ \ve{p}(\ve{r}_{j})/{\alpha_{\rm
          CM}}(\ve{r}_{j})\right]^* \right\} \right].  \ee
%%%%%%%%%%%%%%%%%%%%%%%%%%%%%%%%%%%%%%%%%%%%%%%%
where $u$ or $v$, stand for either $x$ ,$y$, or $z$. The symbol $*$
denotes the complex conjugate. $\ve{r}^g_{j}$ is the vector bewteen
$j$ and the center of masse of the object.
